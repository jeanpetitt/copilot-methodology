\documentclass{article}
\usepackage{geometry}
\geometry{a4paper, margin=1in}

\title{Research Methodology}
\author{Copilot}
\date{\today}

\begin{document}
\maketitle

\section{Methodology}

\subsection{Research Question and Domain}
This study addresses the research question: \textit{``How to leverage a GPT model for semantic table annotation?''}. Situated at the intersection of several cutting-edge domains, the research draws from Natural Language Processing, Machine Learning, Data Science, Artificial Intelligence, and Information Retrieval. The process entails developing methods for enhancing semantic understanding in table annotation tasks.

\subsection{Research Type}
The study adopts a quantitative research approach. This involves the application of statistical analyses alongside hypothesis testing to evaluate the effectiveness of different GPT model configurations in improving semantic table annotation tasks.

\subsection{Methods}
The investigation centers on two primary methods to utilize the GPT model: prompting techniques and fine-tuning. Within the scope of prompting techniques, we explore:
- \textit{Few-shot prompting}: Here, a limited number of annotated examples are provided to the model to enhance its predictive capability in table annotation tasks.
- \textit{Zero-shot prompting}: This scenario evaluates the model's innate ability to perform semantic table annotation without prior exposure to labeled examples.

In addition, fine-tuning the GPT-3 model is conducted to optimize its performance specifically for our tasks, integrating domain-specific data to tailor its understanding and predictive capability.

\subsection{Participants}
The research involves various experimental setups to test the model's capabilities under the defined methods. There are no human participants; instead, the model itself undergoes experimental tuning and testing.

\subsection{Tools and Datasets}
Several tools and datasets are employed in this research, including:

\begin{itemize}
    \item \textbf{Metrics}: To quantify the model's performance, we use precision, recall, and F1-score.
    \item \textbf{Datasets}: The primary dataset used is SuperSemTab2024. This dataset offers a comprehensive benchmark for evaluating semantic table annotation tasks.
    \item \textbf{Tasks}: Our evaluations are structured around several task components: Column Entity Annotation (CEA), Column Type Annotation (CTA), Column Property Annotation (CPA), Raw Annotation (RA), and Table Topic Detection (TD).
\end{itemize}

\subsection{Procedures}
The methodologies are applied without a specific procedural timeline as the research is exploratory within its quantitative framework. The initial phases involve model configuration and iterative testing, using benchmarks from SuperSemTab2024 to calibrate the model's performance metrics, continuous assessment with emphasis on precision, recall, and F1-score, followed by analytical comparison of few-shot and zero-shot prompting outcomes. Fine-tuning iterations are based on performance insights derived from these evaluations.

This methodology, therefore, allows for an empirical examination of the capabilities of GPT models in the semantic annotation of tables, offering insights into optimizing these models for enhanced information retrieval and processing.

\end{document}

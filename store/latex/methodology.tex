\documentclass{article}
\usepackage{geometry}
\geometry{a4paper, margin=1in}

\title{Research Methodology}
\author{Copilot}
\date{\today}

\begin{document}
\maketitle

\section{Methodology}

This section discusses the methodology employed in addressing the research question: \textit{How can the GPT model be leveraged for semantic table annotation?} The methodology utilized is quantitative, supported by statistical analysis and hypothesis testing. The study is set within the domains of Artificial Intelligence, Data Science, Natural Language Processing, Machine Learning, and Information Retrieval.

\subsection{Research Design}

A quantitative approach is adopted to systematically evaluate the effectiveness of the GPT model for semantic table annotation through a rigorous framework of statistical analysis. This approach facilitates the objective measurement of outcomes and supports the derivation of conclusions based on empirical data.

\subsection{Participants}

The study does not involve specific human participants but rather experimental data in the form of datasets that were subjected to analysis. This data-driven approach enables clear measurement and assessment of the research variables.

\subsection{Tools and Datasets}

The primary tools for this research include various computational frameworks used for conducting experiments. The evaluation metrics employed to assess the performance of the GPT model include precision, recall, and F1-score, providing a comprehensive assessment of the model's accuracy and effectiveness.

The dataset utilized in this research is the \textbf{superSemtab2024} dataset, which is an established dataset in the domain of semantic table annotation. This dataset provides a robust framework for evaluating the potential of GPT models in this context.

\subsection{Experimental Procedures}

\subsubsection{Data Preparation}
To prepare the dataset for experimentation, preprocessing steps were applied, including data cleaning and normalization to ensure consistency and accuracy throughout the experiments. This step involved removing any irrelevant or redundant data that could bias the results.

\subsubsection{Model Deployment}
The GPT model was deployed in a controlled experimental setting where its capacity to perform semantic table annotation was rigorously tested. This involved feeding the preprocessed datasets into the model and obtaining the resultant annotations for analysis.

\subsubsection{Evaluation Metric Calculation}
For each experimental run, precise computation of the precision, recall, and F1-score metrics was performed to evaluate the model's predictive performance. These metrics provided insight into the specific strengths and weaknesses of the model in correctly and efficiently annotating semantic data tables.

\subsection{Statistical Analysis}

Statistical analysis was conducted to test the null hypothesis that the GPT model would not significantly improve semantic table annotation compared to baseline methods. This involved utilizing hypothesis testing frameworks to analyze the performance metrics obtained from different experimental runs. Statistical significance was determined based on the p-values calculated in relation to established significance thresholds.

This methodology section outlines the detailed process and analytical strategies employed to address the research question, providing a solid foundation for the subsequent results and discussion sections of the paper.

\end{document}
